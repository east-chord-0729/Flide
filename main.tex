\documentclass[aspectratio=169]{beamer}

% 테마 설정
\usetheme{Madrid}
\usecolortheme{default}

% 한국어 지원
\usepackage[utf8]{inputenc}
\usepackage{kotex}

% 수학 기호
\usepackage{amsmath}
\usepackage{amssymb}

% 코드 하이라이팅
\usepackage{listings}
\usepackage{xcolor}

% 코드 스타일 설정
\lstset{
    language=Python,
    basicstyle=\ttfamily\small,
    keywordstyle=\color{blue},
    commentstyle=\color{green!60!black},
    stringstyle=\color{red},
    numbers=left,
    numberstyle=\tiny,
    numbersep=5pt,
    frame=single,
    breaklines=true,
    breakatwhitespace=true
}

% 제목 정보
\title{Beamer를 활용한 LaTeX 프레젠테이션}
\subtitle{효과적인 발표 자료 작성법}
\author{발표자 이름}
\institute{소속 기관}
\date{\today}

\begin{document}

% 제목 슬라이드
\begin{frame}
    \titlepage
\end{frame}

% 목차 슬라이드
\begin{frame}{목차}
    \tableofcontents
\end{frame}

\section{Beamer 소개}

\begin{frame}{Beamer란?}
    \begin{itemize}
        \item<1-> LaTeX 기반의 프레젠테이션 클래스
        \item<2-> 수학 공식과 코드를 아름답게 표현
        \item<3-> 다양한 테마와 색상 테마 제공
        \item<4-> PDF 형식으로 출력되어 호환성 우수
    \end{itemize}
\end{frame}

\begin{frame}{Beamer의 장점}
    \begin{columns}
        \begin{column}{0.5\textwidth}
            \textbf{장점:}
            \begin{itemize}
                \item 수학 공식 지원
                \item 코드 하이라이팅
                \item 일관된 디자인
                \item 버전 관리 용이
            \end{itemize}
        \end{column}
        \begin{column}{0.5\textwidth}
            \textbf{단점:}
            \begin{itemize}
                \item 학습 곡선
                \item 실시간 편집 어려움
                \item 복잡한 레이아웃 제한
            \end{itemize}
        \end{column}
    \end{columns}
\end{frame}

\section{수학 공식 예제}

\begin{frame}{수학 공식 표현}
    \begin{block}{이차방정식}
        $ax^2 + bx + c = 0$의 해는 다음과 같습니다:
        \[x = \frac{-b \pm \sqrt{b^2 - 4ac}}{2a}\]
    \end{block}
    
    \pause
    
    \begin{block}{적분 예제}
        \[\int_{0}^{\infty} e^{-x^2} dx = \frac{\sqrt{\pi}}{2}\]
    \end{block}
\end{frame}

\begin{frame}{행렬과 벡터}
    \begin{align}
        A &= \begin{pmatrix}
            a_{11} & a_{12} \\
            a_{21} & a_{22}
        \end{pmatrix} \\
        \vec{v} &= \begin{pmatrix}
            v_1 \\
            v_2
        \end{pmatrix}
    \end{align}
    
    \pause
    
    행렬 곱셈:
    \[A\vec{v} = \begin{pmatrix}
        a_{11}v_1 + a_{12}v_2 \\
        a_{21}v_1 + a_{22}v_2
    \end{pmatrix}\]
\end{frame}

\section{코드 예제}

\begin{frame}[fragile]{Python 코드 예제}
    \begin{lstlisting}
def fibonacci(n):
    """피보나치 수열 계산"""
    if n <= 1:
        return n
    else:
        return fibonacci(n-1) + fibonacci(n-2)

# 피보나치 수열 출력
for i in range(10):
    print(f"F({i}) = {fibonacci(i)}")
    \end{lstlisting}
\end{frame}

\begin{frame}[fragile]{알고리즘 복잡도}
    \begin{block}{시간 복잡도 분석}
        \begin{itemize}
            \item 최선의 경우: $O(1)$
            \item 평균의 경우: $O(n)$
            \item 최악의 경우: $O(2^n)$
        \end{itemize}
    \end{block}
    
    \pause
    
    \begin{alertblock}{주의사항}
        재귀적 피보나치 함수는 큰 수에 대해 매우 비효율적입니다!
    \end{alertblock}
\end{frame}

\section{그래프와 차트}

\begin{frame}{데이터 시각화}
    \begin{center}
        \begin{tabular}{|c|c|c|}
            \hline
            \textbf{알고리즘} & \textbf{시간복잡도} & \textbf{공간복잡도} \\
            \hline
            버블 정렬 & $O(n^2)$ & $O(1)$ \\
            퀵 정렬 & $O(n \log n)$ & $O(\log n)$ \\
            병합 정렬 & $O(n \log n)$ & $O(n)$ \\
            \hline
        \end{tabular}
    \end{center}
\end{frame}

\section{결론}

\begin{frame}{요약}
    \begin{enumerate}
        \item<1-> Beamer는 LaTeX 기반의 강력한 프레젠테이션 도구
        \item<2-> 수학 공식과 코드를 아름답게 표현 가능
        \item<3-> 다양한 테마와 커스터마이징 옵션
        \item<4-> 학술 발표에 매우 적합
    \end{enumerate}
    
    \pause
    
    \begin{center}
        \Large \textbf{감사합니다!}
    \end{center}
\end{frame}

\begin{frame}{질문}
    \begin{center}
        \Huge 질문이 있으시면 언제든지 물어보세요!
        
        \vspace{1cm}
        
        \large 이메일: example@university.edu
    \end{center}
\end{frame}

\end{document}
