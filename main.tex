\documentclass[aspectratio=169]{beamer}

% Setting theme and color
\sloppy
\usetheme{Madrid}
\usecolortheme{default}

\usepackage[utf8]{inputenc}    % UTF-8 encoding support
\usepackage{kotex}             % Korean language package
\usepackage{amsmath}           % Advanced math formatting
\usepackage{amssymb}           % Mathematical symbols
\usepackage{listings}          % Code listing package
\usepackage{xcolor}            % Color package for syntax highlighting
\usepackage{fontspec}          % Font configuration (XeLaTeX compilation required)

% Prevent Beamer from setting its own font configuration and hand over control to fontspec
\usefonttheme{professionalfonts}

% Detailed Pretendard font family configuration
\setsansfont{Pretendard}[
    Path          = ./fonts/,      % Font file path
    Extension     = .otf,          % File extension
    UprightFont   = *-Regular,     % Default (md) series: Regular
    BoldFont      = *-Bold,        % Bold (bf) series: Bold
    FontFace      = {ul}{n}{*-Thin},       % UltraLight (ul) series: Thin
    FontFace      = {el}{n}{*-ExtraLight}, % ExtraLight (el) series: ExtraLight
    FontFace      = {l}{n}{*-Light},       % Light (l) series: Light
    FontFace      = {mb}{n}{*-Medium},     % Medium (mb) series: Medium
    FontFace      = {sb}{n}{*-SemiBold},   % SemiBold (sb) series: SemiBold
    FontFace      = {eb}{n}{*-ExtraBold},  % ExtraBold (eb) series: ExtraBold
    FontFace      = {ub}{n}{*-Black}       % UltraBold/Black (ub) series: Black
]

% Include title page configuration
% Title slide configuration
\title{HQC(Hamming Quasi-Cyclic), the new KEM}
\subtitle{2025 KpqC 연구단 워크숍}
\author{DongHyeon Kim}
\institute{Future cryptography Design Lab (FDL), Kookmin University}
\date{\today}

% Custom title page styling
\setbeamertemplate{title page} {
  \begin{tikzpicture}[remember picture, overlay]
    % Title Display
    \node[text  = black, 
          align = center,
          font  = \fontsize{18}{18}\fontseries{sb}\selectfont] 
      at (current page.center) {\inserttitle};
    
    % Subtitle Display
    \node[text  = black, 
          align = center,
          font  = \fontsize{10}{10}\fontseries{sb}\selectfont] 
      at ([yshift=0.7cm]current page.center) {\insertsubtitle};
    
    % Author DisplayDisplay
    \node[text  = black,
          align = center,
          font  = \fontsize{10}{10}\fontseries{el}\selectfont] 
          at ([yshift=-1cm]current page.center) {
          \insertauthor
    };

    % Institute Display
    \node[text  = black,
          align = center,
          font  = \fontsize{10}{10}\fontseries{el}\selectfont]
      at ([yshift=-1.5cm]current page.center) {\insertinstitute};
    
    % % Date
    % \node[text=blue!50!black] 
    %       at ([yshift=-9cm]current page.center) {
    %       \usebeamerfont{date}\insertdate
    % };
  \end{tikzpicture}
} 

% Include footer configuration
% Footer customization
\setbeamertemplate{footline}{
	\begin{tikzpicture}[remember picture, overlay]
		% Page numbers
		\node[text  = gray,
					align = center,
					font  = \tiny\fontseries{el}\selectfont] 
			at ([yshift=0.4cm]current page.south) {
				\insertframenumber \; / \ \inserttotalframenumber
			};
	\end{tikzpicture}
}

% Remove navigation symbols
\setbeamertemplate{navigation symbols}{} 

% Add required packages for custom title page
\usepackage{tikz}

\begin{document}

\begin{frame}
    \titlepage
\end{frame}

% 목차 슬라이드
\begin{frame}{목차}
    \tableofcontents
\end{frame}

\section{Beamer 소개}

\begin{frame}{Beamer란?}
    \begin{itemize}
        \item<1-> LaTeX 기반의 프레젠테이션 클래스
        \item<2-> 수학 공식과 코드를 아름답게 표현
        \item<3-> 다양한 테마와 색상 테마 제공
        \item<4-> PDF 형식으로 출력되어 호환성 우수
    \end{itemize}
\end{frame}

\begin{frame}{Beamer의 장점}
    \begin{columns}
        \begin{column}{0.5\textwidth}
            \textbf{장점:}
            \begin{itemize}
                \item 수학 공식 지원
                \item 코드 하이라이팅
                \item 일관된 디자인
                \item 버전 관리 용이
            \end{itemize}
        \end{column}
        \begin{column}{0.5\textwidth}
            \textbf{단점:}
            \begin{itemize}
                \item 학습 곡선
                \item 실시간 편집 어려움
                \item 복잡한 레이아웃 제한
            \end{itemize}
        \end{column}
    \end{columns}
\end{frame}

\section{수학 공식 예제}

\begin{frame}{수학 공식 표현}
    \begin{block}{이차방정식}
        $ax^2 + bx + c = 0$의 해는 다음과 같습니다:
        \[x = \frac{-b \pm \sqrt{b^2 - 4ac}}{2a}\]
    \end{block}
    
    \pause
    
    \begin{block}{적분 예제}
        \[\int_{0}^{\infty} e^{-x^2} dx = \frac{\sqrt{\pi}}{2}\]
    \end{block}
\end{frame}

\begin{frame}{행렬과 벡터}
    \begin{align}
        A &= \begin{pmatrix}
            a_{11} & a_{12} \\
            a_{21} & a_{22}
        \end{pmatrix} \\
        \vec{v} &= \begin{pmatrix}
            v_1 \\
            v_2
        \end{pmatrix}
    \end{align}
    
    \pause
    
    행렬 곱셈:
    \[A\vec{v} = \begin{pmatrix}
        a_{11}v_1 + a_{12}v_2 \\
        a_{21}v_1 + a_{22}v_2
    \end{pmatrix}\]
\end{frame}

\section{코드 예제}

\begin{frame}[fragile]{알고리즘 복잡도}
    \begin{block}{시간 복잡도 분석}
        \begin{itemize}
            \item 최선의 경우: $O(1)$
            \item 평균의 경우: $O(n)$
            \item 최악의 경우: $O(2^n)$
        \end{itemize}
    \end{block}
    
    \pause
    
    \begin{alertblock}{주의사항}
        재귀적 피보나치 함수는 큰 수에 대해 매우 비효율적입니다!
    \end{alertblock}
\end{frame}

\section{그래프와 차트}

\begin{frame}{데이터 시각화}
    \begin{center}
        \begin{tabular}{|c|c|c|}
            \hline
            \textbf{알고리즘} & \textbf{시간복잡도} & \textbf{공간복잡도} \\
            \hline
            버블 정렬 & $O(n^2)$ & $O(1)$ \\
            퀵 정렬 & $O(n \log n)$ & $O(\log n)$ \\
            병합 정렬 & $O(n \log n)$ & $O(n)$ \\
            \hline
        \end{tabular}
    \end{center}
\end{frame}

\section{결론}

\begin{frame}{요약}
    \begin{enumerate}
        \item<1-> Beamer는 LaTeX 기반의 강력한 프레젠테이션 도구
        \item<2-> 수학 공식과 코드를 아름답게 표현 가능
        \item<3-> 다양한 테마와 커스터마이징 옵션
        \item<4-> 학술 발표에 매우 적합
    \end{enumerate}
    
    \pause
    
    \begin{center}
        \Large \textbf{감사합니다!}
    \end{center}
\end{frame}

\begin{frame}{질문}
    \begin{center}
        \Huge 질문이 있으시면 언제든지 물어보세요!
        
        \vspace{1cm}
        
        \large 이메일: example@university.edu
    \end{center}
\end{frame}

\end{document}
